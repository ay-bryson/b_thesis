\section{State of the Art}

\subsection{Certification}
\label{certif}
Certification of a new engine includes a thorough safety analysis as described in \cite{easa_certification_2015}.

Of primary concern for the Rotatives department is the \ac{hpt} disk due to the extreme conditions under which it operates and therefore the risk of Hazardous Engine Effects. The latter is defined to include (among others) \textquote{non-containment of high-energy debris}, \textquote{uncontrolled fire}, \textquote{complete inability to shut the engine down} according to the \ac{easa} and the \ac{faa} \cite{easa_certification_2015, faa_guidance_2007}. A safety analysis must show that Hazardous Engine Effects are expected to occur with a probability no greater than \(10^{-7}\) per flight hour.

Engine parts whose failure is likely to result in Hazardous Engine Effects are labelled Engine Critical Parts \cite{easa_certification_2015}; the \ac{hpt} disk also carries this label. Engine Critical Parts are assigned an Approved Life, which defines the \textquote{mandatory replacement life} \cite{easa_certification_2015} of the part and is measured in Engine Flight Cycles, a flight profile that defines a reference flight mission, corresponding approximately to the average flight for which the engine is expected to be used in service. 

Determining a part's Approved Life, commonly referred to as lifing, is a complex process, the majority of which involves \textquote{defining the duty the part is required to sustain} \cite{corran_lifing_2007}, i.e. the design and refinement of the Engine Flight Cycle. One lifing philosophy is that of \ac{ltfc}, which involves determining the \ac{pscl} by means of statistically-determined safety factors. 

\subsection{SA66 Cycle Counter}
\label{sa66}
surrogate FE model

\subsection{Perseus}
\label{pers}

\subsection{Machine Learning}
Complex input data is manually consolidated into features to enable an easy inference of information.

\subsubsection{Polynomial Regression}

\subsubsection{The Neuron}

\subsubsection{Time Series Classification}


\ac{tsc} involves reading time series data and applying one of a finite number of labels to each instance \cite{fawaz_inceptiontime_2019}. 

The UCR Archive \cite{dau_ucr_2019} is a large collection of datasets released to enable research and offer a benchmark dataset to evaluate newly proposed \ac{dl} approaches. 

 and the greatest breakthroughs in \ac{tsc} have only come about within the past few years since the publication \cite{ismail_fawaz_deep_2019, fawaz_inceptiontime_2019}.


\subsubsection{Time Series Regression}

\subsection{Scope}