
\begin{titlepage}

        % \raisebox{-0.5\height}{\includegraphics[height=3cm]{rolls_royce}}
        \centering
        {\includegraphics[height=2.75cm]{tu-logo}}

        \vspace{2.5cm}
        {\Large \textbf{Machine Learning of Engine Health Monitoring Data: Development of a machine learning model for damage prediction of real flight missions}\par}
        \vspace{1cm}
        {\scshape\Large Technische Universit\"at Berlin \par}
        \vspace{0.2cm}
        % {\large Technische Universit\"at Berlin\par}
        {\large Faklut\"at V f\"ur Verkehrs- und Maschinensysteme\\im Fach Festigkeit und Lebensdauer\par}
        \vspace{1.5cm}
        {\huge\bfseries Bachelorarbeit\par}
        \vspace{0.6cm}
        {\large zur Erlangung des akademischen Grades \\Bachelor of Science (B.Sc.)\\im Studiengang Physikalische Ingenieurwissenschaft\par}
        \vspace{1.2cm}
        \begin{tabular}{>{\bfseries}r l}
            Verfasser & Samuel Oliver Bryson \\
            % Geboren & am 21.12.1991 \\ & in Nottingham, Gro\ss{}britannien \\
            Matrikelnr. & 352667 \\
            Ausgabedatum & 29.01.2020 \\
            Eingereicht am & XX.XX.2020 \\
            Betreuung & Prof. Dr.-Ing. Robert Liebich \\
            & Dr. Stephan Pannier \\
            & Tolga Ya\u{g}c\i{}, M.Eng.\\
            & Tobias Werder, M.Sc.\\
            % Themenstellender Betrieb & Rolls-Royce Deutschland Ltd \& Co KG
        \end{tabular}

\end{titlepage}